%% ICPC/TopCoder/CodeForces 상위랭커 
%% http://k2code.blogspot.kr/2012/01/given-integer-array-and-number-x-find.html

\documentclass{article}

\usepackage{amsmath, amssymb}
\usepackage{fullpage}
\usepackage{listings}% http://ctan.org/pkg/listings
\lstset{
  basicstyle=\ttfamily,
  mathescape
}
\usepackage{times}
\usepackage{url}
\usepackage[colorlinks=true, linkcolor=blue,]{hyperref}
\usepackage{xcolor}
\usepackage{tikz,subfig}
\usetikzlibrary{topaths}
\usetikzlibrary{calc, arrows}
\tikzset{
  treenode/.style = {align=center, inner sep=0pt, text centered,
    font=\ttfamily},
  arn_n/.style = {treenode, circle, draw=black,
    text width=1.5em},% arbre rouge noir, noeud noir
  arn_r/.style = {treenode, circle, red, draw=red, 
    text width=1.5em, very thick},% arbre rouge noir, noeud rouge
  arn_x/.style = {treenode, rectangle, draw=black,
    minimum width=0.5em, minimum height=0.5em}% arbre rouge noir, nil
}

%\newcommand{\list}{\mathfrak{L}}
\newcommand{\Z}{\mathbb{Z}}
\newcommand{\R}{\mathbb{R}}
\newcommand{\dv}{{\mathrm{div}}}
\newcommand{\md}{{\mathrm{mod}}}
\newcommand{\lcm}{{\mathrm{lcm}}}
\newcommand{\M}[2]{{\mathfrak{M}_{#1\times #2}}}
\newcommand{\sm}[2]{{\mathfrak{S}_{#1\times #2}}}
\newcommand{\sgn}{{\mathrm{sgn}}}

\newenvironment{psmallmatrix}
  {\left(\begin{smallmatrix}}
  {\end{smallmatrix}\right)}

\newtheorem{df}{Definition}[section]

\date{March 16, 2018}
\title{The 5\textsuperscript{th} Problem Set}

\begin{document}
\maketitle
\section*{Pre-requisites}
You should do the followings
\begin{enumerate}
\item On a separate and clean paper,  you need to describe your own strategy to solve the problems below, and 
	to justify why your strategy is effective while handling each problem
\item On a new clean paper, transform your strategy into an algorithm, using your own form to express algorithms.
	Further, you need to analyze the total running steps of your algorithm and the required memory amount to finish your algorithm.
	Then express the total costs using Big-O notation.
\item Use the \texttt{Code ocean} (\url{https://codeocean.com}) platform; if necessary, you may invite me using my email address 
\texttt{lightsun.kim@gmail.com}.
\item Time limits for each problem
\begin{itemize}
\item Problem \#1: Within 1.5 hours
\item Problem \#2: Within 5 hours
\item Problem \#3: Within 1.5 hours
\item Problem \#4: Within 2 hours
\end{itemize}
Then you need to prepare two answer codes; one is a C code that you have made within each time limit, and the
other is a C code augmented and fixed from the original code later.
\end{enumerate}

\newpage

\section*{Problem \#1}


Let $\M{n}{m}(F)$ be the set of $n\times m$ matrices over a set $F$ for positive integers $n,m$. 
When we write $A\in\M{n}{m}(F)$, 
 we will write an $n\times m$ matrix  $A=(a_{ij})=
		\begin{psmallmatrix} 
				a_{11}&\cdots&a_{1m}\\ 
				\vdots & \ddots &\vdots\\
				a_{n1} & \cdots & a_{nm}
				\end{psmallmatrix}$ for all real numbers $a_{ij}\in F$.  
For $A\in\M{n}{m}(F)$, let $A^t$ be the transpose matrix of $A$; then $A^t=(a_{ji})$ and thus $A^t$ is an $m\times n$ 
matrix whose element at $(i,j)$ is $a_{ji}$ of $A$.		
A sparse matrix $\sm{n}{m}\subsetneq \M{n}{m}$ is a special case of $n\times m$ matrices, but
it has at most 25\% non-zero elements. For example,  we say that $A\in\sm{4}{4}$ when its non-zero elements are two.


To obtain the transpose matrix $A^t$ of $A\in\sm{n}{m}$, a straightforward method relies on a nested $\mathtt{for}$ loop  
in terms of rows and columns. Accordingly, the straightforward method requires $O(nm)$ run-time complexity; 
More precisely, transposition requires $O(nm)$ integer comparisons and thus one needs to use such a nested $\mathtt{for}$
loop.


Now the mission in this problem is to devise an efficient algorithm to transpose $A$ into $A^t$ with 
$O(n+m)$ run-time complexity. To this end, you may need to use additional memory spaces.


\bigskip
\noindent\textbf{Language requirements. }%
During tackling this problem, you should follow the programming rules:
\begin{itemize}
\item You should use an ANSI C programming language whose source code can run on \texttt{Code ocean} platform. 
\item Function naming: Begin with the lower character, and every parameters are strong-typed variables (i.e., do not use \texttt{void} typed variables).
	All functions should have a single return value; thus even if a function will return no values; you should provide \texttt{return} keyword.
\item Variable naming: Begin with a type-discriminating prefix. For example, if a variable name is for an age and is with an integer type,
	you need to declare the variable as \texttt{int iAge;}  Especially for string-type variables you are strongly recommended to use the prefix \texttt{sz}.
	For example, if a variable name is for a name, then \texttt{szName} is a preferable choice.
\end{itemize}


\bigskip
\noindent\textbf{Input format.} %
The input is given a text-format file, named \texttt{input.txt} and all strings are separated by blanks.
The file contains the dimension $(n,m)$ of matrix and an indicator $T$ for a base set $F$ by which 
the data type of the input matrix is determined. %We mentioned it as a base set $F$. 
As an example, $T$ can be chosen from a predefined list 
$(\mathtt{Z},\mathtt{Q},\mathtt{R},\mathtt{C},\ldots)$ where $\mathtt{Z}$ means the integers, $\mathtt{Q}$ 
means the rational numbers and so on. Of course, you can specify a customized set such as a set $\mathtt{V}$
whose elements are positive and even numbers.
In addition, the file should specify all $n\times m$ elements of a sparse matrix $A\in\sm{n}{m}$; thus the input file is given in form as follows:
\begin{lstlisting}[backgroundcolor=\color{yellow!40}]
(n m)
T
$a_{11}\ a_{12}\ \ldots\ a_{1m}$
$a_{21}\ a_{22}\ \ldots\ a_{2m}$
$\cdots$
$a_{n1}\ a_{n2}\ \ldots\ a_{nm}$
\end{lstlisting}



\bigskip
\noindent\textbf{Output format.} %
The output should be given as a text-format file, named \texttt{output.txt}.
The output file writes the transpose matrix $A^t\in\sm{m}{n}$ of $A\in\sm{n}{m}$. 
Moreover, you are required to implement two algorithms; one is a basic algorithm whose complexity is $O(mn)$ and 
the other is its improved version whose complexity is $O(n+m)$. 
For this reason, a comparison of execution times between two algorithms should be given beneath the transpose matrix.


\begin{lstlisting}[backgroundcolor=\color{yellow!40}]
$a_{11}\ a_{21}\ \ldots\ a_{m1}$
$a_{12}\ a_{22}\ \ldots\ a_{m2}$
$\cdots$
$a_{1n}\ a_{2n}\ \ldots\ a_{mn}$
**************************************
O(n^2): ____ msec vs. O(n): ____ msec
**************************************
\end{lstlisting}


 
 
\newpage
\section*{Problem \#2}

Let $[1,n]:=\{1,2,\ldots,n\}$ for convenience.
Let $S_n$ be the set of all permutations $\sigma$ over  where a bijective function from 
$[1,n]$ to $[1,n]$ is a permutation. That is,
\begin{equation*}
S_n=\left\{\sigma\big|\text{$\sigma:[1,n]\rightarrow[1,n]$ is a bijection}\right\}
\end{equation*}
Then we can see that $|S_n|=n!$. For some $\sigma\in S_n$, usually we would write
\begin{equation*}
\sigma=\begin{pmatrix}
		1 & 2 & \cdots & n\\
		\sigma(1) & \sigma(2) & \cdots & \sigma(n) 
		\end{pmatrix},
\end{equation*}
but from now on, we use $\sigma=(i_1,i_2,\ldots, i_k)\in S_n$ for some $k\leq n$ to denote
a permutation defined by
\begin{equation*}
\begin{cases}
\sigma(i_j)=i_{j+1}, & \text{$j=1,2,\ldots,k-1$}\\
\sigma(i_k)=i_1, & \\
\sigma(\ell)=\ell, & \text{$\ell\not\in\{i_1,i_2,\ldots,i_k\}$}
\end{cases}
\end{equation*}
For example, $\sigma=(1,3,4,5)\in S_6$ means 
\begin{equation*}
\sigma = \begin{pmatrix}
			1 & 2 & 3 & 4 & 5 & 6\\
			3 & 2 & 4 & 5 & 1 & 6
		\end{pmatrix}.
\end{equation*}
Hereafter we call $(i_1,i_2,\ldots,i_k)\in S_n,$  \textbf{$k$-cycle} and in particular, when $k=2$ we call $(i_1,i_2)$ 
\textbf{transposition}. In fact, since $\sigma(i_1)=i_2$ and $\sigma(i_k)=i_1$,
it is obvious that these permutations are cyclic.
Furthermore, we can easily check that if $\sigma$ is a transposition, then $\sigma\circ\sigma=id$ and 
$\sigma^{-1}=\sigma$ where $\circ$ is the composition of functions
and $id$ is the identity function.
For example $(1,2)\circ (1,2)=id$.

Interestingly, all permutations can be written as a composition of disjoint cycles. 
For example, $\sigma=\begin{psmallmatrix} 1 & 2 &3 &4 &5 &6 &7 &8 &9 \\
											3 &9& 6 &5 &7 &1 &4 &8 &2\end{psmallmatrix}.$
Then because it has $1 \mapsto 3\mapsto 6\mapsto 1$, there is a 3-cycle $(1,3,6)$, 
similarly because it has $4\mapsto 5\mapsto 7\mapsto 4$ we can find another 3-cycle $(4,5,7)$.
Further, we can find a 2-cycle $(2,9)$ and that the permutation fixes 8. 
Hence, $\sigma=(1,3,6)\circ(4,5,7)\circ(2,9).$
Generally speaking, since any permutation in $S_n$ is a composition of cycles and 
any cycle is a composition of transpositions, any permutation in $S_n$ is a composition of transpositions.
For example, consider $\sigma^\ast=(1,5,2,4,3)$. Then two expressions for $\sigma^\ast$ as a composition of transpositions are
\begin{equation*}
\sigma^\ast=(1,5)\circ(5,2)\circ(2,4)\circ(4,3)
\end{equation*}
and
\begin{equation*}
\sigma^\ast=(1,2)\circ(3,4)\circ(2,3)\circ(1,2)\circ(2,3)\circ(3,4)\circ(4,5)\circ(3,4)\circ(2,3)\circ(1,2).
\end{equation*}

For an arbitrary permutation $\sigma\in S_n$, we can write it as
a composition of $r$ transpositions,
\begin{equation*}
\sigma=(i_1,i_2,\ldots,i_k)=\tau_1\circ\tau_2\circ\cdots\circ\tau_r
\end{equation*}
where the $\tau_i$'s are transpositions and $r$ is the number of transpositions.
Although the $\tau_i$'s are not determined uniquely, $r \mod 2$ is determined uniquely 
(if $r\mod 2=0$, then $r$ is even; otherwise $r$ is odd). 
For instance, the two expressions for $\sigma^\ast=(1,5,2,4,3)$ as above involve $4$ and $10$ transpositions, 
which are both even because $4\mod 2=10\mod 2=0$. This fact makes it possible to define the sign function for a
permutation as follows.



\begin{df}[sign]
Let $S_n$ be defined as above. Given $\sigma\in S_n$, we define the function $\sgn:S_n\rightarrow \{\pm 1\}$  by
\begin{equation*}
\sgn(\sigma)=(-1)^r
\end{equation*}
when the permutation $\sigma$ is the composition of $r$ transpositions.
We call the function $\sgn$ signum or \textbf{sign} of the permutation $\sigma$.
\end{df}


Now let us turn back to the main point. This problem interests in computing the determinants of arbitrary square matrices 
in $\M{n}{n}(F)$.
Given a square matrix $A=(a_{ij})\in\M{n}{n}(F)$ with each $a_{ij}\in F$ for a base set $F$, 
the determinant of $A$ is defined as a function $\det(A):\M{n}{n}(F)\rightarrow F$ and denoted by $\det(A)$.
More concretely,  for a given matrix $A\in\M{n}{n}(F)$ we define the determinant of $A$  by 
\begin{equation*}
\det(A)=\sum_{\sigma\in S_n}\sgn(\sigma)\cdot a_{\sigma(1),1}\cdot a_{\sigma(2),2}\cdots a_{\sigma(n),n}
\end{equation*}
For example, consider a matrix $B=\begin{psmallmatrix} 2 & 3 \\ 1 & 2 \end{psmallmatrix}\in\M{2}{2}(\Z)$
where $F=\Z$. 
Then since $\sigma_1=(1,2),\sigma_2=(2,1)$, we see that 
\begin{align*}
\det(B)&=\sum_{\sigma\in S_2}\sgn(\sigma)a_{\sigma(1),1}a_{\sigma(2),2}\\
	&=\sgn(\sigma_1)a_{1,1}a_{2,2}+\sgn(\sigma_2)a_{2,1}a_{1,2}\\
	&=(-1)^0\cdot 2\cdot 2+(-1)^1\cdot 3\cdot 1\\
	&=1.
\end{align*}

\bigskip
\noindent\textbf{Language requirements. }%
During tackling this problem, you should follow the programming rules:
\begin{itemize}
\item You should use an ANSI C programming language whose source code can run on \texttt{Code ocean} platform. 
\item Function naming: Begin with the lower character, and every parameters are strong-typed variables (i.e., do not use \texttt{void} typed variables).
	All functions should have a single return value; thus even if a function will return no values; you should provide \texttt{return} keyword.
\item Variable naming: Begin with a type-discriminating prefix. For example, if a variable name is for an age and is with an integer type,
	you need to declare the variable as \texttt{int iAge;}  Especially for string-type variables you are strongly recommended to use the prefix \texttt{sz}.
	For example, if a variable name is for a name, then \texttt{szName} is a preferable choice.
\end{itemize}


\bigskip
\noindent\textbf{Input format.} %
The input is given a text-format file, named \texttt{input.txt} and all strings are separated by blanks.
The file contains the dimension $(n,n)$ of matrix and an indicator $T$ for a base set $F$ by which 
the data type of the input matrix is determined. 
As mentioned before, you can specify a customized set such as a set $\mathtt{V}$
whose elements are positive and even numbers.
In addition, the file should specify all $n^2$ elements of $A\in\M{n}{m}(F)$; thus the input file is given in form as follows:
\begin{lstlisting}[backgroundcolor=\color{yellow!40}]
(n n)
T
$a_{11}\ a_{12}\ \ldots\ a_{1n}$
$a_{21}\ a_{22}\ \ldots\ a_{2n}$
$\cdots$
$a_{n1}\ a_{n2}\ \ldots\ a_{nn}$
\end{lstlisting}



\bigskip
\noindent\textbf{Output format.} %
The output should be given as a text-format file, named \texttt{output.txt}.
The output file writes the determinant of $A\in$ along with all elements.


\begin{lstlisting}[backgroundcolor=\color{yellow!40}]
$\lceil a_{11}\ a_{21}\ \ldots\ a_{m1}\ \rceil$
$|a_{12}\ a_{22}\ \ldots\ a_{m2}\ |$
$|\cdots\qquad\qquad\qquad\ |$ = $\det(A)$
$|\cdots\qquad\qquad\qquad\  |$ 
$\lfloor a_{1n}\ a_{2n}\ \ldots\ a_{mn}\rfloor$

\end{lstlisting}


% 
\newpage
\section*{Problem \#3} 

In this problem we deal with a quite  simple problem that is almost never related with mathematics.
Let $L$ be a sorted list of $n$ distinct integers; that is, $L=(a_1,a_2,\ldots,a_n)$ where 
each $a_i\in[1,n+1]$. Then it is clear that in the list $L$, there is exactly one element $a\in[1,n+1]$ but 
$a\not\in L$. Now design an efficient algorithm to 
find the integer $a$ in this range that is not in $L$ while having $O(\log n)$ run-time complexity.
However, as before in Problem \#1 you are required to present a naive algorithm with $O(n)$ 
run-time complexity.

\bigskip
\noindent\textbf{Language requirements. }%
During tackling this problem, you should follow the programming rules:
\begin{itemize}
\item You should use an ANSI C programming language whose source code can run on \texttt{Code ocean} platform. 
\item Function naming: Begin with the lower character, and every parameters are strong-typed variables (i.e., do not use \texttt{void} typed variables).
	All functions should have a single return value; thus even if a function will return no values; you should provide \texttt{return} keyword.
\item Variable naming: Begin with a type-discriminating prefix. For example, if a variable name is for an age and is with an integer type,
	you need to declare the variable as \texttt{int iAge;}  Especially for string-type variables you are strongly recommended to use the prefix \texttt{sz}.
	For example, if a variable name is for a name, then \texttt{szName} is a preferable choice.
\end{itemize}


\bigskip
\noindent\textbf{Input format.} %
The input is given a text-format file, named \texttt{input.txt} and all strings are separated by blanks.
The file first describes the number of elements $n>1000$ and an unsorted list $L^\ast=(a_1,\ldots,a_n)$.
If $n\leq 10^3$, then abort the program. The order of each element $a_i$ is randomly selected during writing 
it at the file and a missing element also should be randomly picked up.



\begin{lstlisting}[backgroundcolor=\color{yellow!40}]
n
L*=$\{a_{1}\ a_{2}\ \ldots\ a_{n}\}$
\end{lstlisting}



\bigskip
\noindent\textbf{Output format.} %
The output should be given as a text-format file, named \texttt{output.txt}.
The output file writes the sorted list $L$ and the finding result $a$ in different lines.
In addition, you need to report on a comparison result between your $O(n)$ algorithm and $O(\log n)$ algorithm
in milli-seconds. 
During sorting the unsorted list $L^\ast$, you need to implement the $\mathtt{quicksort}$ algorithm but 
your implementation should be an \textbf{iterative} version of the $\mathtt{quicksort}$ algorithm.
Notice that do not copy any kinds of $\mathtt{quicksort}$ algorithms publicly open in the Internet; 
instead refer to a proper textbook that describes the $\mathtt{quicksort}$ algorithm.
Of course, sorting is not the main part of this problem.



\begin{lstlisting}[backgroundcolor=\color{yellow!40}]
L=$(a_{1}\ a_{2}\ \ldots\ a_{n})$

****************************************
O(n): ____ msec vs. O(log n): ____ msec
****************************************
\end{lstlisting}

% 
\newpage
\section*{Problem \#4} 

This problem has a similar setting as in problem \#3.
However, this problem considers a concrete scenario as follows.
Your mission is a part of automating a medicine prescription management system for 
an imaginary company, named $\mathtt{Abc}$. 
More specifically, When a prescription order comes into the automating system,
it is given as a sequence of $\ell$ medication requests,
\begin{equation*}
\text{$a_1$ oz of drug $X_1$, $a_2$ oz of drug $X_2$},\ldots,\text{$a_\ell$ oz of drug $X_\ell$}
\end{equation*}
where $a_i\in\R$ (the reals) and $X_i$ is the name of medication. Examples of $X_i$'s include Acetaminophen,
Chrolpheniramine maleate, or Pseudoephedrine hydrochloride. 
We assume that $a_1<a_2<\cdots <a_\ell$.


In reality, $\mathtt{Abc}$ has an unlimited supply of $n$ distinctly sized empty bottles for each medication.
Each bottle has a tag specifying its capacity in ounces (for example, $4.23$ oz and $134$ oz). 
To process a medication order, you need to match each medication \textbf{request}--``$a_i$ oz of drug $X_i$''
with the \textbf{size} of the smallest bottle in the inventory 
than can contain $a_i$ ounces.
Therefore you need to design an algorithm to process such a medication order of $\ell$ requests 
while requiring at most $O(\ell\log(\frac{n}{\ell}))$ run-time complexity.
Assume that the bottle sizes are stored in an array $B$, sorted by their capacities in ounces.

\bigskip
\noindent\textbf{Language requirements. }%
During tackling this problem, you should follow the programming rules:
\begin{itemize}
\item You should use an ANSI C programming language whose source code can run on \texttt{Code ocean} platform. 
\item Function naming: Begin with the lower character, and every parameters are strong-typed variables (i.e., do not use \texttt{void} typed variables).
	All functions should have a single return value; thus even if a function will return no values; you should provide \texttt{return} keyword.
\item Variable naming: Begin with a type-discriminating prefix. For example, if a variable name is for an age and is with an integer type,
	you need to declare the variable as \texttt{int iAge;}  Especially for string-type variables you are strongly recommended to use the prefix \texttt{sz}.
	For example, if a variable name is for a name, then \texttt{szName} is a preferable choice.
\end{itemize}


\bigskip
\noindent\textbf{Input format.} %
The input is given a text-format file, named \texttt{input.txt} and all strings are separated by blanks.
The file  describes the sequence of $\ell$ pairs of requests, $(a_i,X_i)_{i\in[1,\ell]}$ and an sorted list $B=(b_1,b_2,\ldots,b_n)$.
As described above, all $a_i$'s are the real numbers.



\begin{lstlisting}[backgroundcolor=\color{yellow!40}]
$(a_1,X_1)\ (a_2,X_2)\ \ldots \ (a_\ell,X_\ell)$ 
$(b_1\ b_2\ \ldots\ b_n)$
\end{lstlisting}



\bigskip
\noindent\textbf{Output format.} %
The output should be given as a text-format file, named \texttt{output.txt}.
The output file writes a list of triples of processing results, $(b_j,a_i,X_i)$ for 
some $j\in[1,n]$ and $i\in[1,\ell]$.


\begin{lstlisting}[backgroundcolor=\color{yellow!40}]
$(b_{j_1},a_1,X_1)\ (b_{j_2},a_2,X_2)\ \ldots\ (b_{j_\ell},a_\ell,X_\ell)\ $
\end{lstlisting}

%
%Roughly speaking, there are dozens of traversal methods of a binary tree. Some of well-known tree traversals include pre-order, post-order, 
%in-order. 
%An important application of tree traversal is to evaluate an arithmetic express $E$ without considering parentheses included in $E$.
%For example, consider an arithmetic expression
%\begin{align*}
%E=((((3+1)\ast 3)/((9-5)+2))-((3\ast(7-4))+6)).
%\end{align*}
%This expression can be represented as a proper binary tree as follows.
%\begin{figure}[h]
%\centering
%\begin{tikzpicture}[level 1/.style={sibling distance=50mm}, level 2/.style={sibling distance=25mm}, level 3/.style={sibling distance=15mm}, 
%	level 4/.style={sibling distance=10mm}, level distance = 10mm]
%
%\node[arn_n]  (a1) {$-$}
%	child {node[arn_n] (a2) {$/$}
%		child{node[arn_n] (a3) {$\ast$}
%			child{node[arn_n] (a7) {$+$}
%				child{node[arn_n] (a9) {$3$}}
%				child{node[arn_n] (a10){$1$}}
%			}
%			child{node[arn_n] (a8) {$3$}
%			}
%		}
%		child{node[arn_n] (a4) {$+$}
%			child{node[arn_n] (a11) {$-$}
%				child{node[arn_n] (a12) {$9$}}
%				child{node[arn_n] (a19) {$5$}}
%			}
%			child{node[arn_n] (a5) {$2$}}
%		}
%	}
%	child{node[arn_n] (a6) {$+$}
%		child{node[arn_n] (a13) {$\ast$}
%			child{node[arn_n] (a15) {$3$}}
%			child{node[arn_n] (a16) {$-$}
%				child{node[arn_n] (a17) {$7$}}
%				child{node[arn_n] (a18) {$4$}}
%			}
%		}
%		child{node[arn_n] (a14) {$6$}}
%	};
%%\draw[->,bend right =15,thick,red] (a9.east) to (a7.south);
%\end{tikzpicture}
%\end{figure} 
%
%Then evaluating the arithmetic express $E$ by the inorder traversal is performed as follows:
%\begin{figure}[h]
%\centering
%\begin{tikzpicture}[level 1/.style={sibling distance=50mm}, level 2/.style={sibling distance=25mm}, level 3/.style={sibling distance=15mm}, 
%	level 4/.style={sibling distance=10mm}, level distance = 10mm]
%
%\node[arn_n]  (a1) {$-$}
%	child {node[arn_n] (a2) {$/$}
%		child{node[arn_n] (a3) {$\ast$}
%			child{node[arn_n] (a7) {$+$}
%				child{node[arn_n] (a9) {$3$}}
%				child{node[arn_n] (a10){$1$}}
%			}
%			child{node[arn_n] (a8) {$3$}
%			}
%		}
%		child{node[arn_n] (a4) {$+$}
%			child{node[arn_n] (a11) {$-$}
%				child{node[arn_n] (a12) {$9$}}
%				child{node[arn_n] (a19) {$5$}}
%			}
%			child{node[arn_n] (a5) {$2$}}
%		}
%	}
%	child{node[arn_n] (a6) {$+$}
%		child{node[arn_n] (a13) {$\ast$}
%			child{node[arn_n] (a15) {$3$}}
%			child{node[arn_n] (a16) {$-$}
%				child{node[arn_n] (a17) {$7$}}
%				child{node[arn_n] (a18) {$4$}}
%			}
%		}
%		child{node[arn_n] (a14) {$6$}}
%	};
%\draw[->,bend right =15,thick,red] (a9.east) to (a7.260);
%\draw[->,bend right =15,thick,red] (a7.280) to (a10.west);
%\draw[->,bend right =15,thick,red] (a10.north) to (a3.260);
%\draw[->,bend right =15,thick,red] (a3.280) to (a8.west);
%\draw[->,bend right =15,thick,red] (a8.north) to (a2.260);
%\draw[->,thick,red] (a2.280) to[out=290,in=100] (a12.north);
%\draw[->,bend right =15,thick,red] (a12.east) to (a11.260);
%\draw[->,bend right =15,thick,red] (a11.280) to (a19.west);
%\draw[->,bend right =15,thick,red] (a19.north) to (a4.260);
%\draw[->,bend right =15,thick,red] (a4.280) to (a5.west);
%\draw[->,bend left =15,thick,red] (a5.north) to (a1.260);
%\draw[->,bend left =15,thick,red] (a1.280) to (a15.north);
%\draw[->,bend right =15,thick,red] (a15.east) to (a13.260);
%\draw[->,bend right =15,thick,red] (a13.280) to (a17.north);
%\draw[->,bend right =15,thick,red] (a17.east) to (a16.260);
%\draw[->,bend right =15,thick,red] (a16.280) to (a18.west);
%\draw[->,bend right =15,thick,red] (a18.north) to (a6.260);
%\draw[->,bend right =15,thick,red] (a6.280) to (a14.west);
%\end{tikzpicture}
%\end{figure} 
%
%To the contrary, this problem considers an operation in this opposite direction.
%This is to say, given a binary tree $T$ representing an arithmetic expression
%you are required to print the expression $E$ by reading the tree $T$.
%One popular way to do this task is to use a tree traversal known as the \textbf{Euler tour traversal}.
%
%Roughly, the Euler tour traversal of a binary tree $T$ can be considered as a ``walk'' around the tree $T$,
%where start by going from the root $r$ towards its left child $T_L(r)$, thinking of the edges of $T$ as being 
%``walks'' that you always keep to your left. Refer to Figure \ref{fig:euler}, in the next page. 
%Yo can see that you stop by each node $v\in T$ three times by the Euler tour which consists of three basic operations:
%\begin{enumerate}
%\item Left visit. You will see $v$ before the Euler tour of $v$'s left subtree $T_L(v)$. Denote it by $\mathtt{left}(v)$.
%\item Below visit. You will also see $v$ immediately after the Euler tour of $v$'s two subtrees). Denote it by $\mathtt{below}(v)$.
%\item Right visit. You will see again $v$ after the Euler tour of $v$'s right subtree $T_R(v)$. Denote it by  $\mathtt{right}(v)$.
%\end{enumerate}
%
%You may observe that if $v$ is an external node, then those three visiting operations all happen at the same time.
%Thus the Euler tour traversal is a generalization of pre-order, in-order, and post-order traversals. 
%In other words, the pre-order traversal of $T$ is equivalent to an Euler tour traversal such that
%each node has an associated ``visit'' operation occur only when 
%it is encountered on the left.
%Similarly, the in-order and post-order traversals of $T$ are equivalent to an Euler tour traversal such that 
%each node has an associated ``visit'' operation occur only when it is encountered from below or on the right, respectively.
% 
%\begin{figure}[h]
%\centering
%\begin{tikzpicture}[level 1/.style={sibling distance=50mm}, level 2/.style={sibling distance=25mm}, level 3/.style={sibling distance=15mm}, 
%	level 4/.style={sibling distance=10mm}, level distance = 10mm,
%	%scale=0.7, transform shape
%]
%
%\node[arn_n]  (a1) {$-$}
%	child {node[arn_n] (a2) {$/$}
%		child{node[arn_n] (a3) {$\ast$}
%			child{node[arn_n] (a7) {$+$}
%				child{node[arn_n] (a9) {$3$}}
%				child{node[arn_n] (a10){$1$}}
%			}
%			child{node[arn_n] (a8) {$3$}
%			}
%		}
%		child{node[arn_n] (a4) {$+$}
%			child{node[arn_n] (a11) {$-$}
%				child{node[arn_n] (a12) {$9$}}
%				child{node[arn_n] (a19) {$5$}}
%			}
%			child{node[arn_n] (a5) {$2$}}
%		}
%	}
%	child{node[arn_n] (a6) {$+$}
%		child{node[arn_n] (a13) {$\ast$}
%			child{node[arn_n] (a15) {$3$}}
%			child{node[arn_n] (a16) {$-$}
%				child{node[arn_n] (a17) {$7$}}
%				child{node[arn_n] (a18) {$4$}}
%			}
%		}
%		child{node[arn_n] (a14) {$6$}}
%	};
%\draw[->, blue, rounded corners, dashed, line width=1pt] 
%($(a1)+(-0.4,0.3)$) -- ($(a2) +(-0.3,0.4)$) -- ($(a3) +(-0.6,0.1)$) -- ($(a7)  +(-0.4,0.3)$) -- 
%($(a9)  +(-0.5,0.0)$) -- ($(a9)  +(-0.4,-0.35)$) -- ($(a9)  +(0.0,-0.5)$) -- ($(a9)  +(0.4,-0.35)$) --  ($(a9)  +(0.5,0.0)$) --  ($(a7)  +(0.0,-0.4)$) --
%($(a10)  +(-0.5,0.0)$) --  ($(a10)  +(-0.4,-0.35)$) -- ($(a10)  +(0.0,-0.5)$) --  ($(a10)  +(0.4,-0.35)$) --  ($(a10)  +(0.5,0.0)$) --
%($(a7)  +(0.5,-0.1)$) --  ($(a3)  +(0.0,-0.7)$) -- ($(a8)  +(-0.5,0.0)$) -- ($(a8)  +(-0.4,-0.35)$) -- ($(a8)  +(0.0,-0.5)$) --
%($(a8)  +(0.4,-0.35)$) -- ($(a8)  +(0.5,0.0)$) -- ($(a3)  +(0.5,-0.1)$) -- ($(a2)  +(0.0,-0.5)$) -- ($(a4)  +(-0.6,-0.1)$) --
%($(a11)  +(-0.2,0.4)$) -- ($(a11)  +(-0.45,0.0)$) -- ($(a11)  +(-0.35,-0.2)$) -- ($(a12)  +(0.0,0.5)$) -- ($(a12)  +(-0.25,0.4)$) --
%($(a12)  +(-0.5,0.0)$) -- ($(a12)  +(-0.4,-0.35)$) -- ($(a12)  +(0.0,-0.5)$) -- ($(a12)  +(0.4,-0.35)$) -- ($(a12)  +(0.5,0.0)$) --
%($(a11)  +(0.0,-0.4)$) -- ($(a19)  +(-0.5,0.0)$) --
%($(a19)  +(-0.5,0.0)$) -- ($(a19)  +(-0.4,-0.35)$) -- ($(a19)  +(0.0,-0.5)$) -- ($(a19)  +(0.4,-0.35)$) -- ($(a19)  +(0.5,0.0)$) --
%($(a11)  +(0.5,-0.1)$) --  ($(a4)  +(0.0,-0.7)$) -- ($(a5)  +(-0.5,0.0)$) -- ($(a5)  +(-0.4,-0.35)$) -- ($(a5)  +(0.0,-0.5)$) --
%($(a5)  +(0.4,-0.35)$) -- ($(a5)  +(0.5,0.0)$) -- ($(a4)  +(0.4,0.2)$) -- ($(a2)  +(0.7,0.0)$) --
%($(a1)  +(0.0,-0.3)$) -- ($(a6)  +(-0.7,0.0)$) -- ($(a13)  +(-0.4,0.2)$) -- ($(a15)  +(-0.4,0.2)$) --
%($(a15)  +(-0.5,0.0)$) --  ($(a15)  +(-0.4,-0.35)$) -- ($(a15)  +(0.0,-0.5)$) --  ($(a15)  +(0.4,-0.35)$) --  ($(a15)  +(0.5,0.0)$) --
% ($(a13)  +(0.0,-0.6)$) -- ($(a16)  +(-0.4,0.3)$) -- 
%($(a17)  +(-0.5,0.0)$) -- ($(a17)  +(-0.4,-0.35)$) -- ($(a17)  +(0.0,-0.5)$) -- ($(a17)  +(0.4,-0.35)$) --  ($(a17)  +(0.5,0.0)$) --  ($(a16)  +(0.0,-0.4)$) --
%($(a18)  +(-0.5,0.0)$) --  ($(a18)  +(-0.4,-0.35)$) -- ($(a18)  +(0.0,-0.5)$) --  ($(a18)  +(0.4,-0.35)$) --  ($(a18)  +(0.5,0.0)$) --
%($(a16)  +(0.4,0.2)$) --   ($(a13)  +(0.5,0.0)$) -- ($(a6)  +(0.0,-0.4)$) -- ($(a14) +(-0.1,-0.4)$) -- ($(a14) +(0.1,-0.4)$) --
%($(a14) + (0.3,-0.3)$) -- ($(a14) + (0.5,0.0)$) --  ($(a14) + (0.4,0.2)$) -- ($(a6) + (0.15,0.4)$) --  ($(a1) + (0.5,0.25)$) ;
%\end{tikzpicture}
%\caption{Euler tour traversal of a binary tree}\label{fig:euler}
%\end{figure}  
%
%\bigskip
%The goal in this problem is simply to implement a C program to print a fully parenthesized arithmetic expression $E$ from its 
%representing binary tree $T$. Your program should implement and invoke the three Euler tour operations $\mathtt{left}(),\mathtt{below}()$, and $\mathtt{right}()$, properly.
%Assume that you have only six arithmetic operations $+,-,{\ast},/$, {\textasciicircum}, and $\%$.
%
%\bigskip
%\noindent\textbf{Input format.} %
%The input is given a text-format file, named \texttt{input.txt}. This file describes a binary tree $T$ of $N$ nodes in the level-order manner where each node $v$ is 
%written by $v=[u_{i1},u_{i2},u_{i3}]$ for the label name of $v$ and its left/right children $u_{i2},u_{i3}$, respectively.
%Indicating a node without children is by \#.
%\begin{lstlisting}[backgroundcolor=\color{yellow!40}]
%[$u_{11}$,$u_{12}$,$u_{13}$]
%[$u_{21}$,$u_{22}$,$u_{23}$]
%$\cdots$
%[$u_{N1}$,$u_{N2}$,$u_{N3}$]
%\end{lstlisting}
%For example,  the binary tree in Figure~\ref{fig:euler} is given by
%\begin{lstlisting}[backgroundcolor=\color{white}]
%[-,/,+]
%[/,*,+]
%[+,*,6]
%[*+,3]
%[+,-,2]
%[*,3,-]
%[6,#,#]
%[+,3,1]
%[3,#,#]
%[-,9,3]
%[2,#,#]
%[3,#,#]
%[-,7,4]
%[3,#,#]
%[1,#,#]
%[9,#,#]
%[5,#,#]
%[7,#,#]
%[4,#,#]
%\end{lstlisting}
%
%
%\bigskip
%\noindent\textbf{Output format.} %
%The output should be given as a text-format file, named \texttt{outputc.txt}, \texttt{outputq.txt}, or \texttt{outputl.txt}.
%The output file writes a fully parenthesized arithmetic expression.
%
%\begin{lstlisting}[backgroundcolor=\color{yellow!40}]
%$\ast\ast$ The arithmetic expression $\ast\ast$
%$E=$
%\end{lstlisting}
%
%\newpage
%\begin{thebibliography}{100}
%%\addtolength{\leftmargin}{0.2in}
%%\setlength{\itemindent}{-0.2in}
%
%\bibitem[HSAf08]{HSAf08} E. Horowitz, S. Sahni, and S. Anderson-Free, \emph{Fundamentals of Data Structures in C}, Second edition, Silicon Press, 2008. 
%
%
%%\bibitem[Jos27]{Jos27}  Flavius Josephus, \emph{The jewish war  Book III},  Translated by H. S. Thackeray,  Heinemann 1927,
%%342--366 \& 387--391.
% \end{thebibliography}            
\end{document}

 


































